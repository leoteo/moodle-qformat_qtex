\newcommand{\toNemFileNoArg}[1]{}
\newcommand{\toNemFile}[2]{}
\newcommand{\generateNemFile}{%
    \newwrite\nemesisWrite%
    \immediate\openout\nemesisWrite=\jobname.nem%
    \newtoks\nemesisToks%
    \renewcommand{\toNemFileNoArg}[1]{%
        \immediate\write\nemesisWrite{##1}%
        \immediate\write\nemesisWrite{0}%
    }%
    \renewcommand{\toNemFile}[2]{%
        \immediate\write\nemesisWrite{##1}%
        \immediate\write\nemesisWrite{1}%
        \nemesisToks={##2}%
        \immediate\write\nemesisWrite{\the\nemesisToks}%
    }%
}

\newcommand{\keepme}[1]{\toNemFile{keepme}{#1}{#1}}
\newcommand{\intro}[1]{\toNemFile{intro}{#1}{#1}}

\newcounter{questionOrdinal}
\setcounter{questionOrdinal}{0}

\newcounter{answerOrdinal}
\setcounter{answerOrdinal}{0}

\newcommand{\question}[1]{%
    \addtocounter{questionOrdinal}{1}%
    \setcounter{answerOrdinal}{0}%
    \toNemFile{question}{#1}%
    \styleQuestion{#1}%
}

\newcommand{\questionSc}[1]{%
    \addtocounter{questionOrdinal}{1}%
    \setcounter{answerOrdinal}{0}%
    \toNemFile{questionSc}{#1}%
    \styleQuestionSc{#1}%
}

\newcommand{\true}[1]{%
    \addtocounter{answerOrdinal}{1}%
    \toNemFile{true}{#1}%
    \styleTrue{#1}%
}

\newcommand{\false}[1]{%
    \addtocounter{answerOrdinal}{1}%
    \toNemFile{false}{#1}%
    \styleFalse{#1}%
}

\newcommand{\feedback}[1]{%
    \toNemFile{feedback}{#1}%
    \styleFeedback{#1}%
}

\newcommand{\explanation}[1]{%
    \toNemFile{explanation}{#1}%
    \styleExplanation{#1}%
}

\newcommand{\onlyt}[1]{%
    \addtocounter{answerOrdinal}{1}%
    \toNemFile{onlyt}{#1}%
    \styleTrue{#1}%
}

\newcommand{\onlyf}[1]{%
    \addtocounter{answerOrdinal}{1}%
    \toNemFile{onlyf}{#1}%
    \styleFalse{#1}%
}

\newcommand{\dunno}{%
    \toNemFileNoArg{dunno}
    \styleDunno%
}

\newcommand{\hidesolution}{%
    \renewcommand{\feedback}[1]{%
        \toNemFile{feedback}{##1}%
    }%
    \renewcommand{\explanation}[1]{%
        \toNemFile{explanation}{##1}%
    }%
    \renewcommand{\styleTrue}{%
        \styleTrueHidden%
    }%
    \renewcommand{\styleFalse}{%
        \styleFalseHidden%
    }%
    \renewcommand{\styleDunno}{%
        \styleDunnoHidden%
    }%
}

\newcommand{\tags}[1]{}
\newcommand{\authors}[1]{}
\newcommand{\locale}[2]{}
\newcommand{\category}[2]{}

\newcommand{\questionBrowser}[1]{%
    \addtocounter{questionOrdinal}{1}%
    \setcounter{answerOrdinal}{0}%
    \styleQuestionBrowser{#1}%
}

\newcommand{\questionBrowserSc}[1]{%
    \addtocounter{questionOrdinal}{1}%
    \setcounter{answerOrdinal}{0}%
    \styleQuestionBrowserSc{#1}%
}

\newcommand{\answerUp}[1]{%
    \addtocounter{answerOrdinal}{1}%
    \styleAnswerUp{#1}%
}

\newcommand{\solutionQuestion}[1]{%
    \addtocounter{questionOrdinal}{1}%
    \setcounter{answerOrdinal}{0}%
    \styleSolutionQuestion{#1}%
}

\newcommand{\solutionQuestionSc}[1]{%
    \addtocounter{questionOrdinal}{1}%
    \setcounter{answerOrdinal}{0}%
    \styleSolutionQuestionSc{#1}%
}

\newcommand{\trueSol}[1]{%
    \addtocounter{answerOrdinal}{1}%
    \styleTrueSol{#1}%
}

\newcommand{\falseSol}[1]{%
    \addtocounter{answerOrdinal}{1}%
    \styleFalseSol{#1}%
}

\newcommand{\solutionExplanation}[1]{%
    \styleSolutionExplanation{#1}%
}

\newcommand{\solutionFeedback}[1]{%
    \styleSolutionFeedback{#1}%
}

%%
 % #1 userId, i.e. key value
 % #2 sentStamp
 % #3 score
 % #4 potentialScore
 %%
\newcommand{\introSubmissionMin}[4]{%
    \styleIntroSubmissionMin{#1}{#2}{#3}{#4}%
}

\newcommand{\trueChecked}[1]{%
    \addtocounter{answerOrdinal}{1}%
    \styleTrueChecked{#1}%
}

\newcommand{\trueUnchecked}[1]{%
    \addtocounter{answerOrdinal}{1}%
    \styleTrueUnchecked{#1}%
}

\newcommand{\falseChecked}[1]{%
    \addtocounter{answerOrdinal}{1}%
    \styleFalseChecked{#1}%
}

\newcommand{\falseUnchecked}[1]{%
    \addtocounter{answerOrdinal}{1}%
    \styleFalseUnchecked{#1}%
}

\newcommand{\answerDown}[1]{%
    \addtocounter{answerOrdinal}{1}%
    \styleAnswerDown{#1}%
}

\newcommand{\answerDunnoDown}{%
    \styleAnswerDunnoDown%
}
\newcommand{\styleQuestion}[1]{#1}
\newcommand{\styleQuestionSc}[1]{#1}
\newcommand{\styleTrue}[1]{#1}
\newcommand{\styleFalse}[1]{#1}
\newcommand{\styleFeedback}[1]{#1}
\newcommand{\styleExplanation}[1]{#1}
\newcommand{\styleDunno}{}

\newcommand{\styleTrueHidden}[1]{#1}
\newcommand{\styleFalseHidden}[1]{#1}
\newcommand{\styleDunnoHidden}{}

\newcommand{\styleQuestionBrowser}[1]{#1}
\newcommand{\styleQuestionBrowserSc}[1]{#1}
\newcommand{\styleUpBrowser}[1]{#1}
\newcommand{\styleDunnoUpBrowser}{}

\newcommand{\styleAnswerUp}[1]{#1}

\newcommand{\styleSolutionQuestion}[1]{#1}
\newcommand{\styleSolutionQuestionSc}[1]{#1}

\newcommand{\styleTrueSol}[1]{#1}
\newcommand{\styleFalseSol}[1]{#1}
\newcommand{\styleSolutionExplanation}[1]{#1}
\newcommand{\styleSolutionFeedback}[1]{#1}

\newcommand{\styleIntroSubmissionMin}[4]{#1 #2 #3 #4}

\newcommand{\styleTrueChecked}[1]{#1}
\newcommand{\styleTrueUnchecked}[1]{#1}
\newcommand{\styleFalseChecked}[1]{#1}
\newcommand{\styleFalseUnchecked}[1]{#1}

\newcommand{\styleSolutionDunnoChecked}{}
\newcommand{\styleSolutionDunnoUnchecked}{}

\newcommand{\styleAnswerDown}[1]{#1}
\newcommand{\styleAnswerDunnoDown}{}
\renewcommand{\styleQuestion}[1]{%
    \bigskip%
    \filbreak%
    \noindent {\bf\arabic{questionOrdinal}.\ }{#1}%
}

\renewcommand{\styleQuestionSc}{\styleQuestion}

\renewcommand{\styleQuestionBrowser}[1]{%
    \filbreak%
    \noindent {\bf\arabic{questionOrdinal}.\ }{#1}%
}

\renewcommand{\styleQuestionBrowserSc}{\styleQuestionBrowser}

\renewcommand{\styleTrue}[1]{%
    \begin{itemize}%
        \item[\begin{tabular}{rr}$\surd$&(\alph{answerOrdinal})\end{tabular}]{#1}
    \end{itemize}%
}

\renewcommand{\styleFalse}[1]{%
    \begin{itemize}%
        \item[\begin{tabular}{rr}&(\alph{answerOrdinal})\end{tabular}]{#1}
    \end{itemize}%
}

\newcommand{\dunnoText}{Weiss ich nicht.}

\renewcommand{\styleDunno}{%
    \begin{itemize}%
        \item[]{\dunnoText}%
    \end{itemize}%
}

\renewcommand{\styleFeedback}[1]{%
    \begin{itemize}%
        \item[]{\par{\footnotesize{#1}}}%
    \end{itemize}%
}

\renewcommand{\styleFalseHidden}[1]{%
    \begin{itemize}%
        \item[\begin{tabular}{rr}&(\alph{answerOrdinal})\end{tabular}]{#1}
    \end{itemize}%
}

\renewcommand{\styleUpBrowser}[1]{%
    \addtocounter{answerOrdinal}{1}%
    \styleFalseHidden{#1}%
}

\renewcommand{\styleDunnoUpBrowser}{%
    \begin{itemize}%
        \item[]{\dunnoText}%
    \end{itemize}%
}

\renewcommand{\styleTrueHidden}[1]{%
    \begin{itemize}%
        \item[\begin{tabular}{rr}&(\alph{answerOrdinal})\end{tabular}]{#1}
    \end{itemize}%
}

\renewcommand{\styleDunnoHidden}{%
    \begin{itemize}%
        \item[]{\dunnoText}%
    \end{itemize}%
}

\renewcommand{\styleAnswerUp}{\styleFalseHidden}
\renewcommand{\styleAnswerDown}[1]{%
    \begin{itemize}%
        \item[%
            \begin{tabular}{lc}%
                & \fbox{(\alph{answerOrdinal})}%
            \end{tabular}%
            ]%
            {#1}%
    \end{itemize}%
}

\renewcommand{\styleSolutionQuestion}[1]{%
    \bigskip%
    \filbreak%
    \noindent {\bf\arabic{questionOrdinal}.\ }%
    {\scriptsize {#1}}%
}

\renewcommand{\styleSolutionQuestionSc}{\styleSolutionQuestion}

\renewcommand{\styleTrueSol}[1]{%
    \begin{itemize}%
        \item[%
            \begin{tabular}{rr}%
                $\surd$&(\alph{answerOrdinal})%
            \end{tabular}%
            ]%
            {\scriptsize {#1}}
    \end{itemize}%
}

\renewcommand{\styleFalseSol}[1]{%
    \begin{itemize}%
        \item[%
            \begin{tabular}{rr}%
                &(\alph{answerOrdinal})%
            \end{tabular}%
            ]%
            {\scriptsize {#1}}
    \end{itemize}%
}

\renewcommand{\styleSolutionExplanation}[1]{%
    \par\noindent {#1}%
}

\renewcommand{\styleSolutionFeedback}[1]{%
    \begin{itemize}%
        \item[]{\par{#1}}%
    \end{itemize}%
}

\renewcommand{\styleIntroSubmissionMin}[4]{%
    \begin{center}%
        \Large{\tt L\"osung}%
    \end{center}%
%
    \texttt{#3 / #4} \\
    \texttt{#2}      \\
    \texttt{#1}

\bigskip\hrule\bigskip%
}

\renewcommand{\styleTrueChecked}[1]{%
    \begin{itemize}%
        \item[%
            \begin{tabular}{lc}%
                $\surd$ & \fbox{(\alph{answerOrdinal})}%
            \end{tabular}%
            ]%
                {\scriptsize #1}%
    \end{itemize}%
}

\renewcommand{\styleTrueUnchecked}[1]{%
    \begin{itemize}%
        \item[%
            \begin{tabular}{lc}%
                $\surd$ & (\alph{answerOrdinal})%
            \end{tabular}%
            ]%
                {\scriptsize #1}%
    \end{itemize}%
}

\renewcommand{\styleFalseChecked}[1]{%
    \begin{itemize}%
        \item[%
            \begin{tabular}{lc}%
                & \fbox{(\alph{answerOrdinal})}%
            \end{tabular}%
            ]%
                {\scriptsize #1}%
    \end{itemize}%
}

\renewcommand{\styleFalseUnchecked}[1]{%
    \begin{itemize}%
        \item[%
            \begin{tabular}{lc}%
                & (\alph{answerOrdinal})%
            \end{tabular}%
            ]%
                {\scriptsize #1}%
    \end{itemize}%
}

\renewcommand{\styleSolutionDunnoChecked}{%
    \begin{itemize}
        \item[%
            \begin{tabular}{lc}%
                & $\bigotimes$%
            \end{tabular}%
            ]%
                {\scriptsize \dunnoText}%
    \end{itemize}%
}

\renewcommand{\styleSolutionDunnoUnchecked}{%
}

\renewcommand{\styleAnswerDunnoDown}{%
    \begin{itemize}
        \item[%
            \begin{tabular}{lc}%
                & $\bigotimes$%
            \end{tabular}%
            ]%
                {\scriptsize \dunnoText}%
    \end{itemize}%
}
\newcommand{\nemRatio}[2]{%
    $\displaystyle\frac{#1}{#2}$%
}
%%
 % #1 actual
 % #2 potential
 % #3 prozent(actual/potential)
 % #4 available points
 % #5 max points
 % #6 min points
 % #7 arithmetic mean
 %%
\newcommand{\introSolutionTeacherZwei}[7]{%
    \begin{center}%
    \begin{tabular}{lrr}
        {\tt Beteiligung}                 & \nemRatio{#1}{#2} & #3\% \cr%
                                                                     \cr%
        {\tt Erreichbare Punktzahl}       & $#4$              &      \cr%
        {\tt Maximal erreichte Punktzahl} & $#5$              &      \cr%
        {\tt Minimal erreichte Punktzahl} & $#6$              &      \cr%
        {\tt Arithmetisches Mittel}       & $#7$              &      \cr%
    \end{tabular}%
    \end{center}%
    \bigskip%
    \hrule%
    \bigskip%
    \bigskip%
}
%%
 % #1 actual
 % #2 potential
 % #3 prozent(actual/potential)
 % #4 available points
 % #5 max points
 % #6 min points
 % #7 arithmetic mean
 % #8 own points
 %%
\newcommand{\introSolutionStudentZwei}[8]{%
    \begin{center}%
    \begin{tabular}{lrr}
        {\tt Beteiligung}                 & \nemRatio{#1}{#2} & #3\% \cr%
                                                                     \cr%
        {\tt Erreichbare Punktzahl}       & $#4$              &      \cr%
                                                                     \cr%
        {\tt Ihre Punktzahl}              & $#8$              &      \cr%
                                                                     \cr%
        {\tt Maximal erreichte Punktzahl} & $#5$              &      \cr%
        {\tt Minimal erreichte Punktzahl} & $#6$              &      \cr%
        {\tt Arithmetisches Mittel}       & $#7$              &      \cr%
    \end{tabular}%
    \end{center}%
    \bigskip%
    \hrule%
    \bigskip%
    \bigskip%
}

\newcommand{\studentId}[4]{%
    \begin{center}%
        #1-#2-#3, #4%
    \end{center}%
}%

\newcommand{\tutorialIds}[1]{%
    \begin{center}%
        {\tt {#1}}%
    \end{center}%
}%
%%
 % #1 lecture
 % #2 actualHandover
 % #3 potentialHandover
 % #4 maxScore
 % #5 minScore
 % #6 averageScore
 %%
\newcommand{\introSolutionTeacher}[6]{%
    \styleIntroSolutionTeacher{#1}{#2}{#3}{#4}{#5}{#6}%
}

\newcommand{\styleIntroSolutionTeacher}[6]{{#1}{#2}{#3}{#4}{#5}{#6}}

\renewcommand{\styleIntroSolutionTeacher}[6]{%
    \begin{description}%
        \item[\texttt{Abgaben:}] $#2$ / $#3$
        \item[\texttt{Maximal erreichte Punktzahl:}] $#4$
        \item[\texttt{Minimal erreichte Punktzahl:}] $#5$
        \item[\texttt{Durchschnitt:}] $#6$
    \end{description}%
    \bigskip
    \hrule
    \bigskip
    \bigskip%
}


%%
 % #1 percentage who did not know the answer
 % #2 percentage who gave exactly the correct answers
 % #3 body text of the question
 %%
\newcommand{\questionAftermathMulti}[3]{%
    \addtocounter{questionOrdinal}{1}%
    \setcounter{answerOrdinal}{0}%
    \styleQuestionAftermathMulti{#1}{#2}{#3}%
}

%%
 % #1 students answer
 % #2 percentage who did not know the answer
 % #3 percentage who gave exactly the correct answers
 % #4 body text of the question
 %%
\newcommand{\questionAftermathMultiStudent}[4]{%
    \addtocounter{questionOrdinal}{1}%
    \setcounter{answerOrdinal}{0}%
    \styleQuestionAftermathMultiStudent{#1}{#2}{#3}{#4}%
}

\newcommand{\styleQuestionAftermathMulti}[3]{{#1}{#2}{#3}}
\newcommand{\styleQuestionAftermathMultiStudent}[4]{{#1}{#2}{#3}{#4}}

\renewcommand{\styleQuestionAftermathMulti}[3]{%
    \bigskip%
    \filbreak%
    \begin{center}%
        \begin{tabular}{rl}%
            #2\% & {\tt Korrekt} \cr%
            #1\% & {\tt Nicht gewusst}
        \end{tabular}
    \end{center}%
    \noindent {\bf\arabic{questionOrdinal}.\ }{#3}%

}

\newcommand{\studentKorrekt}{%
    {\tt Ihre L\"osung war korrekt.}%
}
\newcommand{\studentInkorrekt}{%
    {\tt Ihre L\"osung war nicht korrekt.}%
}
\newcommand{\studentDunno}{%
    {\tt Sie haben die L\"osung nicht gewusst.}%
}

\renewcommand{\styleQuestionAftermathMultiStudent}[4]{%
    \bigskip%
    \filbreak%
    \begin{center}%
        \begin{tabular}{rl}%
                 & #1            \cr%
            #3\% & {\tt Korrekt} \cr%
            #2\% & {\tt Nicht gewusst}
        \end{tabular}
    \end{center}%
    \noindent {\bf\arabic{questionOrdinal}.\ }{#4}%
}

%%
 % #1 percentageAnswer
 % #2 body text of the answer
 %%
\newcommand{\aftermathTrue}[2]{%
    \addtocounter{answerOrdinal}{1}%
    \styleAftermathTrue{#1}{#2}%
}

%%
 % #1 percentageAnswer
 % #2 body text of the answer
 %%
\newcommand{\aftermathTrueChecked}[2]{%
    \addtocounter{answerOrdinal}{1}%
    \styleAftermathTrueChecked{#1}{#2}%
}

\newcommand{\styleAftermathTrue}[2]{{#1}{#2}}
\newcommand{\styleAftermathTrueChecked}[2]{{#1}{#2}}

\renewcommand{\styleAftermathTrue}[2]{%
    \begin{itemize}%
        \item[%
            \begin{tabular}{lrc}%
                $\surd$ & #1\% & (\alph{answerOrdinal})%
            \end{tabular}%
        ]%
            {#2}%
    \end{itemize}%
}

\renewcommand{\styleAftermathTrueChecked}[2]{%
    \begin{itemize}%
        \item[%
            \begin{tabular}{lrc}%
                $\surd$ & #1\% & \fbox{(\alph{answerOrdinal})}%
            \end{tabular}%
        ]%
            {#2}%
    \end{itemize}%
}

%%
 % #1 percentageAnswer
 % #2 body text of the answer
 %%
\newcommand{\aftermathFalse}[2]{%
    \addtocounter{answerOrdinal}{1}%
    \styleAftermathFalse{#1}{#2}%
}

%%
 % #1 percentageAnswer
 % #2 body text of the answer
 %%
\newcommand{\aftermathFalseChecked}[2]{%
    \addtocounter{answerOrdinal}{1}%
    \styleAftermathFalseChecked{#1}{#2}%
}

\newcommand{\styleAftermathFalse}[2]{{#1}{#2}}
\newcommand{\styleAftermathFalseChecked}[2]{{#1}{#2}}

\renewcommand{\styleAftermathFalse}[2]{%
    \begin{itemize}%
        \item[%
            \begin{tabular}{lrc}%
                & #1\% & (\alph{answerOrdinal})%
            \end{tabular}%
        ]%
            {#2}%
    \end{itemize}%
}

\renewcommand{\styleAftermathFalseChecked}[2]{%
    \begin{itemize}%
        \item[%
            \begin{tabular}{lrc}%
                & #1\% & \fbox{(\alph{answerOrdinal})}%
            \end{tabular}%
        ]%
            {#2}%
    \end{itemize}%
}

\newcommand{\aftermathDunno}[1]{%
    \styleAftermathDunno{#1}%
}

\newcommand{\styleAftermathDunno}[1]{{#1}}

\renewcommand{\styleAftermathDunno}[1]{%
    \begin{itemize}%
        \item[%
            \begin{tabular}{lrr}%
                #1\% & &%
            \end{tabular}
        ]%
            \dunnoText%
    \end{itemize}%
}

% ========================================================================
% ========================================================================
% ========================================================================

\newcommand{\nemesisI}{}
\newcommand{\nemesisII}{}
\newcommand{\nemesisIII}{}
\newcommand{\nemesisIV}{}
\newcommand{\nemesisV}{}
\newcommand{\nemesisVI}{}
\newcommand{\nemesisVII}{}
\newcommand{\nemesisVIII}{}
\newcommand{\nemesisIX}{}
%%
 % This is the intro of the aftermath for a tutor with the results of her/his 
 % group. Due to a TeX restriction of at most 9 parameters we have to make a
 % loop. 
 %
 % @param #1 lecture
 % @param #2 userid
 % @param #3 actualHandoverYanaTutor
 % @param #4 potentialHandoverYanaTutor
 % @param #5 actualHandover
 % @param #6 potentialHandover
 % @param #7 maxScoreYanaTutor
 % @param #8 maxScore
 % @param #9 minScoreYanaTutor
 % @param #10 minScore
 % @param #11 averageScoreYanaTutor
 % @param #12 averageScore
 % @param #13 Wann
 % @param #14 Wo
 %%
\newcommand{\introSolutionYanaTutor}[9]{%
     \renewcommand{\nemesisI}{{#1}}%
     \renewcommand{\nemesisII}{{#2}}%
     \renewcommand{\nemesisIII}{{#3}}%
     \renewcommand{\nemesisIV}{{#4}}%
     \renewcommand{\nemesisV}{{#5}}%
     \renewcommand{\nemesisVI}{{#6}}%
     \renewcommand{\nemesisVII}{{#7}}%
     \renewcommand{\nemesisVIII}{{#8}}%
     \renewcommand{\nemesisIX}{{#9}}%
     \introSolutionYanaTutorPre%
}
\newcommand{\introSolutionYanaTutorPre}[5]{%
     \styleIntroSolutionYanaTutorPre{#1}{#2}{#3}{#4}{#5}%
} 
\newcommand{\styleIntroSolutionYanaTutorPre}[5]{%
    \nemesisI \nemesisII \nemesisIII \nemesisIV \nemesisV \nemesisVI%
    \nemesisVII \nemesisVIII \nemesisIX #1 #2 #3 #4 #5%
}

%%
 % This is the question environment for an aftermath yanaGroup vs. lecture 
 % with more than one correct answer
 %
 % @param #1 percentage of participant in the yanaGroup who did not answer the
 %           question
 % @param #2 percentage of participant of the lecture who did not answer the
 %           question
 % @param #3 percentage of participant in the yanaGroup who gave exactly the
 %           correct answers
 % @param #4 percentage of participant of the lecture who gave the exact correct
 %           answers
 % @param #5 body text of the question
 %%
\newcommand{\questionAftermathMultiYana}[5]{%
    \addtocounter{questionOrdinal}{1}%
    \setcounter{answerOrdinal}{0}%
    \styleQuestionAftermathMultiYana{#1}{#2}{#3}{#4}{#5}%
}

\newcommand{\styleQuestionAftermathMultiYana}[5]{%
    #1 #2 #3 #4 #5%
}

%%
 % This is the aftermath of one answer for the assistant or tutor. 
 % It reports on the percentage of participants in the tutor's group vs. in
 % total giving this answer, which is in this case true .
 %
 % @param #1 percentageAnswerYanaTutor
 % @param #2 percentageAnswer
 % @param #3 body text of the answer
 %%
\newcommand{\aftermathYanaTutorTrue}[3]{%
    \addtocounter{answerOrdinal}{1}%
    \styleAftermathYanaTutorTrue{#1}{#2}{#3}%
}

\newcommand{\styleAftermathYanaTutorTrue}[3]{%
    #1 #2 #3%
}

%%
 % This is the aftermath of one answer for the assistant or tutor. 
 % It reports on the percentage of participants in the tutor's group vs. in
 % total giving this answer, which is in this case false.
 %
 % @param #1 percentageAnswerYanaTutor
 % @param #2 percentageAnswer
 % @param #3 body text of the answer
 %%
\newcommand{\aftermathYanaTutorFalse}[3]{%
    \addtocounter{answerOrdinal}{1}%
    \styleAftermathYanaTutorFalse{#1}{#2}{#3}%
}

\newcommand{\styleAftermathYanaTutorFalse}[3]{%
    #1 #2 #3%
}

\newcommand{\aftermathYanaTutorDunno}[2]{%
    \styleAftermathYanaTutorDunno{#1}{#2}%
}

\newcommand{\styleAftermathYanaTutorDunno}[2]{%
    #1 #2%
}

\renewcommand{\styleAftermathYanaTutorTrue}[3]{%
    \begin{itemize}%
        \item[%
            \begin{tabular}{lrc}%
                $\surd$ & Ca. #1\%/#2\% & (\alph{answerOrdinal})%
            \end{tabular}%
        ]%
            {#3}%
    \end{itemize}%
}

\renewcommand{\styleAftermathYanaTutorFalse}[3]{%
    \begin{itemize}%
        \item[%
            \begin{tabular}{lrc}%
                & Ca. #1\%/#2\% & (\alph{answerOrdinal})%
            \end{tabular}%
        ]%
            {#3}%
    \end{itemize}%
}

\renewcommand{\styleAftermathYanaTutorDunno}[2]{%
    \begin{itemize}%
        \item[%
            \begin{tabular}{lrr}%
                #1\% / #2\% & &%
            \end{tabular}%
        ]%
            \dunnoText%
    \end{itemize}%
}

\renewcommand{\styleQuestionAftermathMultiYana}[5]{%
    \bigskip%
    \filbreak%
    \noindent {\bf\arabic{questionOrdinal}.\ (#3\% / #4\%)\ }%
    {#5}%
}

\renewcommand{\styleIntroSolutionYanaTutorPre}[5]{%
    \begin{center}%
        \Large{\bf Auswertung und L\"osung \\ Gruppe:} {\tt \nemesisII} \\
        {\tt \footnotesize #4 #5}%
    \end{center}%
    \smallskip%
    \begin{tabular}{lcc}%
        & \texttt{Ihre Gruppe} & \texttt{Insgesamt} \\
        \texttt{Abgaben:}  & $\nemesisIII$ / $\nemesisIV$ & 
            $\nemesisV$ / $\nemesisVI$ \\
        \texttt{Maximal erreichte Punktzahl:} & $\nemesisVII$ &
            $\nemesisVIII$ \\
        \texttt{Minimal erreichte Punktzahl:} & $\nemesisIX$ & $#1$ \\
        \texttt{Durchschnitt:} & $#2$ & $#3$
    \end{tabular}\\

    \noindent%
Bei den Antworten gibt die erste Prozentzahl jeweils an, wie viele
Studierende Ihrer Gruppe diese Antwort angegeben hatten, die zweite wie viele
insgesamt.
\medskip
\hrule
\bigskip%
}
